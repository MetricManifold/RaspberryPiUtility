Telescope Control System is a C++ program for the Raspberry Pi. The program communicates with a connected Arduino that reorients a telescope. The telescope can be reoriented using the inputted horizontal celestial coordinates from the Qt Interface on the Pi.

\subsection*{Getting Started}

The instructions below will allow you to compile the project and run it on the Raspberry Pi.

\subsubsection*{Prerequisites}

Install the Qt libraries using {\ttfamily apt}\+: 
\begin{DoxyCode}
sudo apt install qt5-default
\end{DoxyCode}


\subsubsection*{Installation}

Generate the {\ttfamily Makefile} using the existing project {\ttfamily telescope.\+pro}\+: 
\begin{DoxyCode}
qmake telescope.pro
\end{DoxyCode}


Build the program\+: 
\begin{DoxyCode}
make
\end{DoxyCode}


\subsection*{Usage}

After installation just run the program from the directory {\ttfamily group25}\+: 
\begin{DoxyCode}
./telescope
\end{DoxyCode}


\subsection*{Development}

If you wish to build the program with the files you are developing, edit the {\ttfamily telescope.\+pro} file to include {\ttfamily .h} and {\ttfamily .cpp} files. The {\ttfamily telescope.\+pro} file should look like this\+: 
\begin{DoxyCode}
QT += widgets

TEMPLATE = app
TARGET = telescope
INCLUDEPATH += .

DEFINES += QT\_DEPRECATED\_WARNINGS


# Input
HEADERS += backendwrapper.h \(\backslash\)
           telescopeui.h
FORMS += raspberrypiutility.ui
SOURCES += backendwrapper.cpp \(\backslash\)
           main.cpp \(\backslash\)
           telescopeui.cpp
RESOURCES += raspberrypiutility.qrc
\end{DoxyCode}


Qt {\ttfamily .ui} files should be appended to {\ttfamily F\+O\+R\+MS}, C++ header {\ttfamily .h} files to {\ttfamily H\+E\+A\+D\+E\+RS}, C++ files {\ttfamily .cpp} to {\ttfamily S\+O\+U\+R\+C\+ES}.

To generate a fresh Qt project file\+: 
\begin{DoxyCode}
qmake -project -o telescope.pro
\end{DoxyCode}
 Edit {\ttfamily telescope.\+pro} to include\+: 
\begin{DoxyCode}
QT += widgets
\end{DoxyCode}
 